\documentclass[11pt,a4paper]{article}

%\usepackage[utf8]{inputenc}
\usepackage[english]{babel}
\usepackage[T1]{fontenc}

\usepackage{amsmath,amssymb,amsfonts}

\begin{document}

\begin{figure}
  \thispagestyle{empty}
  \centering
    \vspace*{0.5cm}
    {\Huge ARK, Group Assignment 1 \par
    }\par
    %\par\vspace*{4\onelineskip}
    \par
    \vspace*{0.4cm}
    \large Jonas Brunsgaard - 141185 \par
        Rasmus Wriedt Larsen - 070290 \par
        Martin Bjeregaard Jepsen \par
    \vspace*{0.4cm}
    Sep 20th 2012 \par
    \vspace*{0.4cm}
    \small Instructor: Thomas Nicolaj Barnholdt \par
    \vspace*{0.4cm}
    \small Department of Computer Science \par
    \small University of Copenhagen \par

\end{figure}
\clearpage
\thispagestyle{plain}

\section*{Introduction}
It is our perception, that this report answers all the questions asked in the
first group assignment of the course. Thus we find the assignment to be fully
answered.

\section*{G1.1 - AND, OR, NOT, XOR}

Verification of the correctness of our solution can be done using truth-tables. We compare the reference
functions to our implementations by outputting a truth table in Logisim (Project > Analyze Circuit > Table).
We can clearly see that all of our implementations are correct.

% NOT

\begin{table}
    \begin{tabular}{l || l || l}
        A & Reference output & Actual output \\ \hline
        1 & 0                & 0             \\
        0 & 1                & 1             \\
    \end{tabular}
\end{table}

% OR

\begin{table}
    \begin{tabular}{l | l || l || l}
        A & B & Reference output & Actual output \\ \hline
        1 & 1 & 1                & 1        \\
        1 & 0 & 1                & 1        \\
        0 & 1 & 1                & 1        \\
        0 & 0 & 0                & 0        \\
    \end{tabular}
\end{table}

% AND

\begin{table}
    \begin{tabular}{l | l || l || l}
        A & B & Reference output & Actual output \\ \hline
        1 & 1 & 1                & 1        \\
        1 & 0 & 0                & 0        \\
        0 & 1 & 0                & 0        \\
        0 & 0 & 0                & 0        \\
    \end{tabular}
\end{table}

% XOR

\begin{table}
    \begin{tabular}{l | l || l || l}
        A & B & Reference output & Actual output \\ \hline
        1 & 1 & 0                & 0        \\
        1 & 0 & 1                & 1        \\
        0 & 1 & 1                & 1        \\
        0 & 0 & 0                & 0        \\
    \end{tabular}
\end{table}

\section*{G1.2 - Building a 4bit ALU}

\end{document}

