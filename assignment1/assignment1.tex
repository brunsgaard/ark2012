\documentclass[11pt,a4paper]{article}

%\usepackage[utf8]{inputenc}
\usepackage[english]{babel}
\usepackage[T1]{fontenc}

\usepackage{amsmath,amssymb,amsfonts}

\begin{document}

\begin{figure}
  \thispagestyle{empty}
  \centering
    \vspace*{0.5cm}
    {\Huge ARK, Group Assignment 1 \par
    }\par
    %\par\vspace*{4\onelineskip}
    \par
    \vspace*{0.4cm}
    \large Jonas Brunsgaard - 141185 \par
        Rasmus Wriedt Larsen - 070290 \par
        Martin Bjerregaard Jepsen - 251190 \par
    \vspace*{0.4cm}
    Sep 20th 2012 \par
    \vspace*{0.4cm}
    \small Instructor: Thomas Nicolaj Barnholdt \par
    \vspace*{0.4cm}
    \small Department of Computer Science \par
    \small University of Copenhagen \par
\end{figure}
\clearpage
\thispagestyle{plain}

\section*{Introduction}
It is our perception, that this report answers all the questions asked in the
first group assignment of the course. Thus we find the assignment to be fully
answered.

\section*{G1.1 - AND, OR, NOT, XOR}

Verification of the correctness of our solution can be done using truth-tables. We compare the reference
functions to our implementations by outputting a truth table in Logisim (Project > Analyze Circuit > Table).
We can clearly see that all of our implementations are correct.

\subsection*{NOT} % (fold)
\label{sub:NOT}

\begin{table}[htb!]
    \centering
    \begin{tabular}{c || c || c}
    $A$ & $\lnot A$ & Actual output \\ \hline
    1 & 0                & 0             \\
    0 & 1                & 1             \\
    \end{tabular}
\end{table}

% subsection NOT (end)

% OR
\subsection*{OR} % (fold)
\label{sub:OR}

\begin{table}[htb!]
    \centering
    \begin{tabular}{c | c || c || c}
        $A$ & $B$ & $A \lor B$ & Actual output \\ \hline
        1   & 1   & 1          & 1        \\
        1   & 0   & 1          & 1        \\
        0   & 1   & 1          & 1        \\
        0   & 0   & 0          & 0        \\
    \end{tabular}
\end{table}

% subsection OR (end)

\subsection*{AND} % (fold)
\label{sub:AND}

\begin{table}[htb!]
    \centering
    \begin{tabular}{c | c || c || c}
        $A$ & $B$ & $A \land B$ & Actual output \\ \hline
        1 & 1 & 1                & 1        \\
        1 & 0 & 0                & 0        \\
        0 & 1 & 0                & 0        \\
        0 & 0 & 0                & 0        \\
    \end{tabular}
\end{table}

% subsection AND (end)

\subsection*{XOR} % (fold)
\label{sub:XOR}

\begin{table}[htb!]
    \centering
    \begin{tabular}{c | c || c || c}
        $A$ & $B$ & $A \oplus B$ & Actual output \\ \hline
        1 & 1 & 0                & 0        \\
        1 & 0 & 1                & 1        \\
        0 & 1 & 1                & 1        \\
        0 & 0 & 0                & 0        \\
    \end{tabular}
\end{table}

% subsection XOR (end)

\section*{G1.2 - Building a 4bit ALU}
We have been building a 4bit ALU, with the following functions

\begin{table}[htb!]
\begin{tabular}{| c | c |}
    \centering
    ALU control lines & Function \\ \hline
    0000 & AND \\
    0001 & OR \\
    0010 & add \\
    0110 & subtract \\
    0111 & set on less than \\
    1100 & NOR \\

\end{tabular}
\end{table}

The final 4-bit ALU is created as a ripple-carry adder, as specified in COD\@.
It utilizes 4 1-bit ALUs built using full adders.

\subsection*{Full adder} % (fold)
\label{sub:Full_adder}

\begin{table}[htb!]
    \centering
    \begin{tabular}{c | c || c || c}
        $A$ & $B$ & $A \oplus B$ & $A + B$ (bitwise addition without carry) \\ \hline
        1 & 1 & 0                & 0        \\
        1 & 0 & 1                & 1        \\
        0 & 1 & 1                & 1        \\
        0 & 0 & 0                & 0        \\
    \end{tabular}
\end{table}

The full adder adds two bits $A$ and $B$, with a carry in and carry out taken into
consideration. We observe that $A \oplus B$ is the same as the bitwise addition
of $A$ and $B$ without carry. Therefore

\begin{equation*}
    Sum = (A \oplus B) \oplus CarryIn
\end{equation*}

We now need to set the $CarryOut$ bit. The adder should only report a
carry if both $A$ and $B$ are asserted, or their sum and the $CarryIn$ is asserted.
This leaves us with the boolean expression

\begin{equation*}
    CarryOut = (A \cdot B) + (Sum \cdot CarryIn)
\end{equation*}

% subsection Full adder (end)

\end{document}

